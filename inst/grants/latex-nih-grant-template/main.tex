%% For NIH grant application %%
%% July 2010 %%
%% Tatsuki Koyama %%

\documentclass[11pt]{article}  %11 or 12 pt

% Packages some may find useful.
\usepackage{graphicx,epsf,pstricks,subfigure,psfrag,rotating}

% New NIH specifications
% font = Arial, Helvetica, Palatino Linotype, or Georgia typeface
% font size 11 or larger
% margin = at least one-half inch
% Use at least one-half inch margins for all pages.  
% No information should appear in the margins, including the PI's name and page numbers.

\renewcommand{\rmdefault}{phv} % Arial
\renewcommand{\sfdefault}{phv} % Arial

\usepackage[width=7.0in, height=9.5in, head=0.0in, foot=0.0in, headsep=0.0in]{geometry}
%% This controls margins.  Can't go over width=7.5in, height=10.0in.  
%% top-bottom margins = (11-height)/2  left-right margins = (8.5-width)/2

\usepackage{setspace} % useful in changing vertical spacing temporarily.

\usepackage{natbib} % more control over how references appear within text.
\bibpunct{[}{]}{,}{n}{}{} % like so.
%% use \citep{ref} instead of \cite{ref} in text.

\usepackage{sectsty} % can change font, size of the section headings.  
\sectionfont      {\fontsize{12pt}{3}\usefont{OT1}{phv}{b}{sc}\selectfont}
\subsectionfont   {\fontsize{11pt}{3}\usefont{OT1}{phv}{b}{n}\selectfont}
\subsubsectionfont{\fontsize{11pt}{3}\usefont{OT1}{phv}{m}{n}\selectfont}

\renewcommand{\thesection}{\Alph{section}} % so that section headings use A B C instead 1 2 3
\renewcommand{\baselinestretch}{1}

\renewcommand\refname{\section{Literature Cited} \vspace{-1em}} 
%% This changes ``Reference'' to ``Literature Cited''.  

\newcommand{\inden}[1]{\mbox{} \hspace{#1} } % Force horizontal spaces.  

% -- % -- % -- % -- % -- %
% -- % -- % -- % -- % -- %
\begin{document}
\pagestyle{empty}

% -- % -- % -- % -- % -- %
% -- % -- % -- % -- % -- %
\section{Specific Aims}
State concisely the goals of the proposed research and summarize the expected outcome(s), 
including the impact that the results of the proposed research will exert on the research field(s) involved.

List succinctly the specific objectives of the research proposed, e.g., to test a stated hypothesis, 
create a novel design, solve a specific problem, challenge an existing paradigm or clinical practice, 
address a critical barrier to progress in the field, or develop new technology.

Specific Aims are limited to one page.

% -- % -- % -- % -- % -- %
\subsection{Specific Aim 1: Develop something amazing.}
\inden{2em}   A.1.1: sub aim 1\\
\inden{2em} A.1.2: sub aim 2\\
\inden{2em}   A.1.3: sub aim 3

% -- % -- % -- % -- % -- %
\subsection{Specific Aim 2: Develop more amazing things.}
This and that

% -- % -- % -- % -- % -- %
%\newpage
\section{Research Strategy}
Organize the Research Strategy in the specified order and using the instructions provided below. 
Start each section with the appropriate section heading . Significance, Innovation, Approach. 
Cite published experimental details in the Research Strategy section and provide the full reference 
in the Bibliography and References Cited section.

% -- % -- % -- % -- % -- %
\subsection{Significance}

% -- % -- % -- % -- % -- %
\begin{itemize}
\item Explain the importance of the problem or critical barrier to progress in the field 
that the proposed project addresses.
\item Explain how the proposed project will improve scientific knowledge, technical capability, 
and/or clinical practice in one or more broad fields.
\item Describe how the concepts, methods, technologies, treatments, services, 
or preventative interventions that drive this field will be changed if the proposed aims are achieved.
\end{itemize}

% -- % -- % -- % -- % -- %
\subsection{Innovation}

% -- % -- % -- % -- % -- %
\begin{itemize}
\item Explain how the application challenges and seeks to shift current research or clinical practice paradigms.
\item Describe any novel theoretical concepts, approaches or methodologies, instrumentation or interventions 
to be developed or used, and any advantage over existing methodologies, instrumentation, or interventions.
\item Explain any refinements, improvements, or new applications of theoretical concepts, 
approaches or methodologies, instrumentation, or interventions.
\end{itemize}

% -- % -- % -- % -- % -- %
\subsection{Approach}

% -- % -- % -- % -- % -- %
\begin{itemize}
\item Describe the overall strategy, methodology, and analyses to be used to accomplish the specific aims of the project. 
Unless addressed separately in Item 15 (Resource Sharing Plan), include how the data will be collected, analyzed, 
and interpreted as well as any resource sharing plans as appropriate.
\item Discuss potential problems, alternative strategies, and benchmarks for success anticipated to achieve the aims.
\item If the project is in the early stages of development, describe any strategy to establish feasibility, 
and address the management of any high risk aspects of the proposed work.
\item Point out any procedures, situations, or materials that may be hazardous to personnel and precautions to be exercised. 
A full discussion on the use of Select Agents should appear in Item 11, below. 
\end{itemize}

% -- % -- % -- % -- % -- %
\bibliographystyle{myrefstyle} %unsrt should work, too.  copy myrefstyle.bst in the same directory as the .tex file.
\bibliography{ref} % Or wherever you keep your .bib file.

% -- % -- % -- % -- % -- %
\end{document}
