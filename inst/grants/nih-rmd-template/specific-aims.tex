\documentclass[11pt,]{article}
\usepackage{lmodern}
\usepackage{setspace}
\setstretch{0.9}
\usepackage{amssymb,amsmath}
\usepackage{ifxetex,ifluatex}
\usepackage{fixltx2e} % provides \textsubscript
\ifnum 0\ifxetex 1\fi\ifluatex 1\fi=0 % if pdftex
  \usepackage[T1]{fontenc}
  \usepackage[utf8]{inputenc}
\else % if luatex or xelatex
  \ifxetex
    \usepackage{mathspec}
  \else
    \usepackage{fontspec}
  \fi
  \defaultfontfeatures{Ligatures=TeX,Scale=MatchLowercase}
    \setmainfont[]{Helvetica}
\fi
% use upquote if available, for straight quotes in verbatim environments
\IfFileExists{upquote.sty}{\usepackage{upquote}}{}
% use microtype if available
\IfFileExists{microtype.sty}{%
\usepackage{microtype}
\UseMicrotypeSet[protrusion]{basicmath} % disable protrusion for tt fonts
}{}
\usepackage[margin=.5in]{geometry}
\usepackage{hyperref}
\hypersetup{unicode=true,
            pdfborder={0 0 0},
            breaklinks=true}
\urlstyle{same}  % don't use monospace font for urls
\usepackage{natbib}
\bibliographystyle{plainnat}
\IfFileExists{parskip.sty}{%
\usepackage{parskip}
}{% else
\setlength{\parindent}{0pt}
\setlength{\parskip}{6pt plus 2pt minus 1pt}
}
\setlength{\emergencystretch}{3em}  % prevent overfull lines
\providecommand{\tightlist}{%
  \setlength{\itemsep}{0pt}\setlength{\parskip}{0pt}}
\setcounter{secnumdepth}{5}

%%% Use protect on footnotes to avoid problems with footnotes in titles
\let\rmarkdownfootnote\footnote%
\def\footnote{\protect\rmarkdownfootnote}


  \title{}
    \author{}
    \date{}
  

%%%%%%%%%%
% personal preamble edits here
%%%%%%%%%%
\pagenumbering{gobble}

% can toggle this for Helvetica
%\usepackage{helvet}
%\renewcommand{\familydefault}{\sfdefault}

% \titlespacing*{\paragraph}{0pt}{2pt}{1em}

% set section numbering/lettering
% tips for seccntformat: https://tex.stackexchange.com/questions/95896/how-to-format-subsection-title-without-packages
\makeatletter
\def\@seccntformat#1{%
  \expandafter\ifx\csname c@#1\endcsname\c@section\else
  \expandafter\ifx\csname c@#1\endcsname\c@paragraph\else
  \csname the#1\endcsname\quad
  \fi\fi}
  
  % stucture for these commands: https://texfaq.org/FAQ-atsigns
  \renewcommand\section{
  \@startsection{section}{1}{\z@}
    {-3.5ex \@plus -1ex \@minus -.2ex}
    {1.0ex \@plus.2ex} %reduce space below section (was 1.5ex)
    {\normalfont\normalsize\bf\uppercase}} %modify font style
    
  \renewcommand\subsection{
  \@startsection{subsection}{2}{\z@}
    {-1.5ex\@plus -1ex \@minus -.2ex}%reduce space above subsection (was -3.25ex)
    {0.5ex \@plus .2ex}%reduce space below subsection (was 1.5ex)
    {\normalfont\normalsize\bf}} %modify font style
    
  \renewcommand\subsubsection{
  \@startsection{subsubsection}{3}{\z@}
    {-1.0ex\@plus -1ex \@minus -.2ex}%reduce space above subsubsection (was -3.25ex)
    {0.5ex \@plus .2ex}%reduce space below subsubsection (was 1.5ex)
    {\normalfont\normalsize\bf}} %modify font style
    
  \renewcommand\paragraph{
  \@startsection{paragraph}{4}{\z@}
    {-0.5ex\@plus -1ex \@minus -.2ex}%reduce space above paragraph (was -3.25ex)
    {-1.5ex \@plus .2ex}%convert space below paragraph to an indent (was 1.5ex)
    {\normalfont\normalsize\bf}} %modify font style    
\makeatother

\renewcommand\thesubsection{\Alph{subsection}.}
\renewcommand\thesubsubsection{\thesubsection\arabic{subsubsection}.}

% reduce spacing at the top of lists
\usepackage{enumitem}
\setlist{topsep = 2pt}

% allow text to wrap around figures
\usepackage{graphicx}
\usepackage{wrapfig}

%%%%%%%%%%

\begin{document}

\hypertarget{specific-aims}{%
\section{SPECIFIC AIMS}\label{specific-aims}}

\hypertarget{background}{%
\subsection{Background}\label{background}}

\hypertarget{trajectory-of-adhd-across-development}{%
\subsubsection{Trajectory of ADHD across
development}\label{trajectory-of-adhd-across-development}}

ADHD typically presents early in life, most often diagnosed and treated
between childhood and adolescence. Less commonly, ADHD can insidiously
emerge in adulthood in individuals with no history of inattention,
hyperactivity, or impulsivity in childhood. Notably, longitudinal data
from the MTA ADHD Treatment Study suggests 90-95\% of ``adult-onset
ADHD'' is better accounted for by comorbid psychopathology, personality
disorders, and/or substance use disorders \emph{misdiagnosed} as ADHD.
The later-onset expression of ADHD thus appears to be a syndrome
distinct from childhood-onset ADHD.

\hypertarget{executive-dysfunction-in-emerging-adulthood}{%
\subsubsection{Executive dysfunction in emerging
adulthood}\label{executive-dysfunction-in-emerging-adulthood}}

Interconnected networks of heteromodal association cortex orchestrating
higher executive cognitive functions are still maturing during college.
It is therefore plausible generalized \emph{executive dysfunction}, not
undiagnosed ADHD per se, is underlying most clinical referrals for ADHD
in college students. We posit executive dysfunction is at the core of
the real-world problems facing college students who self-refer for ADHD
evaluation and treatment, which accentuates the comorbid difficulties
these students face such as problematic drug and alcohol use, emotional
dysregulation, and school failure.

\hypertarget{limited-diagnostic-tools-to-assess-adhd-and-executive-functioning-in-college-students}{%
\subsubsection{Limited diagnostic tools to assess ADHD and executive
functioning in college
students}\label{limited-diagnostic-tools-to-assess-adhd-and-executive-functioning-in-college-students}}

Seeking evaluation and treatment for ADHD for the first time is now
especially salient on \emph{college campuses}, where the use/misuse of
prescription stimulant medication as a cognitive-enhancing drug is
rising faster than the actual incidence of the disorder. The
proliferation of college-onset ADHD, whether real or feigned, or early
or late onset, signifies an increase in the demand to evaluate and treat
ADHD across the lifespan. However, clinicians do not have adequate,
evidence-based psychodiagnostic tools available to evaluate ADHD/EF in
this specific population that grows each year, nor do we have many
objective assessment measures sensitive to \emph{symptom validity} and
\emph{performance validity}. Poor diagnostic tools lead to diagnostic
errors, resulting in wasteful, inappropriate, or potentially even
harmful treatment recommendations. Without better research-derived
clinical measures, we will continue to overdiagnose and treat students
faking ADHD for secondary gain \emph{and} underdiagnose students with
unfeigned ADHD who need medical and/or academic interventions.

\hypertarget{framework}{%
\subsection{Framework}\label{framework}}

Our long-term goal is to develop a targeted, performance-based executive
functioning neuropsychological test battery for evaluating the varied
expressions of ADHD and executive dysfunction in college students
utilizing deep learning analytical methods. The following research aims
are proposed:

\hypertarget{aim-1-develop-a-model-of-executive-dysfunction-predictive-of-adhd-fit-using-deep-learning.}{%
\subsubsection{Aim 1: Develop a model of executive dysfunction
predictive of ADHD fit using deep
learning.}\label{aim-1-develop-a-model-of-executive-dysfunction-predictive-of-adhd-fit-using-deep-learning.}}

The model will be psychometrically optimized and sensitive to
developmental trajectories and feigning ADHD symptoms in college
students.

\hypertarget{aim-2-assess-the-validity-and-generalizability-of-the-adhd-classification-model-fit-on-independent-datasets.}{%
\subsubsection{Aim 2: Assess the validity and generalizability of the
ADHD classification model fit on independent
datasets.}\label{aim-2-assess-the-validity-and-generalizability-of-the-adhd-classification-model-fit-on-independent-datasets.}}

Importantly, to reduce the length of the final neuropsychological
battery, poorly performing tests from Aim 1 will not be retained in Aim
2.

\emph{Sample} Study recruitment will focus on an intentionally
heterogeneous cohort of 300 undergraduate and graduate students at USC
with (a) documented histories of confirmed childhood or adolescent ADHD,
(b) self-reported ``adult-onset ADHD,'' or (c) no history or concern of
ADHD to identify core neurocognitive signatures of childhood-onset
versus adult-onset ADHD.

\emph{Executive Function Measures} We will utilize existing
neuropsychological measures clinicians already use so that our resultant
findings can be implemented into everyday clinical practice. The
training test battery will include performance-based, commonly used
neuropsychological measures of executive functioning from the D-KEFS
(sequencing, inhibition, switching), NAB Attention and Executive
Functions Modules (attentional fluency, concept formation, cognitive
efficiency, everyday attention, judgment), NIH EXAMINER (planning,
cognitive control, fluency), WAIS-IV (processing speed, verbal working
memory), and WMS-IV (visual working memory). Intra-individual
variability, group-level variability, group differences, and
psychometric properties (validity, reliability, sensitivity, and
specificity) will be estimated across core latent traits of executive
functioning and systematically tested for their utility in predicting
ADHD. Poorly performing tests and/or cognitive factors will not be
retained for final model validation and testing.

\emph{Malingering} The study will determine which executive function
measures are optimally sensitive to feigning ADHD for secondary gain in
college students. To do so, each neuropsychological test will have a
single-blinded randomized instruction provided to the student to either
(A) give their best effort or (B) fake having ADHD for secondary gain
(trained on how to do so). Data will be split into three sets: training,
validation, and test. The aggregate training and validation datasets
will allow us to make hard predictions for a new case of the form ``Yes,
this student is faking ADHD,'' or ``No, this student is not faking
ADHD.'' The test set allows us to quantify the uncertainty with which
the prediction is made.

\hypertarget{impact}{%
\paragraph{Impact}\label{impact}}

ADHD remains improperly diagnosed and treated in tens of thousands of
promising college students. The consequences of aberrant diagnosis and
treatment of ADHD during this critical stage of young adult development
can lead to functional impairments in school performance, educational
attainment, career development, social-emotional development, and family
and community life. Thus, the need for evidence-based diagnostic tools
specific to this rapidly growing clinical population is urgently
stronger than ever.

\bibliography{references.bib}


\end{document}
