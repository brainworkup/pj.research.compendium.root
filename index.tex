% Options for packages loaded elsewhere
\PassOptionsToPackage{unicode}{hyperref}
\PassOptionsToPackage{hyphens}{url}
\PassOptionsToPackage{dvipsnames,svgnames,x11names}{xcolor}
%
\documentclass[
  super,
  preprint,
  3p]{elsarticle}

\usepackage{amsmath,amssymb}
\usepackage{lmodern}
\usepackage{iftex}
\ifPDFTeX
  \usepackage[T1]{fontenc}
  \usepackage[utf8]{inputenc}
  \usepackage{textcomp} % provide euro and other symbols
\else % if luatex or xetex
  \usepackage{unicode-math}
  \defaultfontfeatures{Scale=MatchLowercase}
  \defaultfontfeatures[\rmfamily]{Ligatures=TeX,Scale=1}
\fi
% Use upquote if available, for straight quotes in verbatim environments
\IfFileExists{upquote.sty}{\usepackage{upquote}}{}
\IfFileExists{microtype.sty}{% use microtype if available
  \usepackage[]{microtype}
  \UseMicrotypeSet[protrusion]{basicmath} % disable protrusion for tt fonts
}{}
\makeatletter
\@ifundefined{KOMAClassName}{% if non-KOMA class
  \IfFileExists{parskip.sty}{%
    \usepackage{parskip}
  }{% else
    \setlength{\parindent}{0pt}
    \setlength{\parskip}{6pt plus 2pt minus 1pt}}
}{% if KOMA class
  \KOMAoptions{parskip=half}}
\makeatother
\usepackage{xcolor}
\setlength{\emergencystretch}{3em} % prevent overfull lines
\setcounter{secnumdepth}{5}
% Make \paragraph and \subparagraph free-standing
\ifx\paragraph\undefined\else
  \let\oldparagraph\paragraph
  \renewcommand{\paragraph}[1]{\oldparagraph{#1}\mbox{}}
\fi
\ifx\subparagraph\undefined\else
  \let\oldsubparagraph\subparagraph
  \renewcommand{\subparagraph}[1]{\oldsubparagraph{#1}\mbox{}}
\fi

\usepackage{color}
\usepackage{fancyvrb}
\newcommand{\VerbBar}{|}
\newcommand{\VERB}{\Verb[commandchars=\\\{\}]}
\DefineVerbatimEnvironment{Highlighting}{Verbatim}{commandchars=\\\{\}}
% Add ',fontsize=\small' for more characters per line
\usepackage{framed}
\definecolor{shadecolor}{RGB}{241,243,245}
\newenvironment{Shaded}{\begin{snugshade}}{\end{snugshade}}
\newcommand{\AlertTok}[1]{\textcolor[rgb]{0.68,0.00,0.00}{#1}}
\newcommand{\AnnotationTok}[1]{\textcolor[rgb]{0.37,0.37,0.37}{#1}}
\newcommand{\AttributeTok}[1]{\textcolor[rgb]{0.40,0.45,0.13}{#1}}
\newcommand{\BaseNTok}[1]{\textcolor[rgb]{0.68,0.00,0.00}{#1}}
\newcommand{\BuiltInTok}[1]{\textcolor[rgb]{0.00,0.23,0.31}{#1}}
\newcommand{\CharTok}[1]{\textcolor[rgb]{0.13,0.47,0.30}{#1}}
\newcommand{\CommentTok}[1]{\textcolor[rgb]{0.37,0.37,0.37}{#1}}
\newcommand{\CommentVarTok}[1]{\textcolor[rgb]{0.37,0.37,0.37}{\textit{#1}}}
\newcommand{\ConstantTok}[1]{\textcolor[rgb]{0.56,0.35,0.01}{#1}}
\newcommand{\ControlFlowTok}[1]{\textcolor[rgb]{0.00,0.23,0.31}{#1}}
\newcommand{\DataTypeTok}[1]{\textcolor[rgb]{0.68,0.00,0.00}{#1}}
\newcommand{\DecValTok}[1]{\textcolor[rgb]{0.68,0.00,0.00}{#1}}
\newcommand{\DocumentationTok}[1]{\textcolor[rgb]{0.37,0.37,0.37}{\textit{#1}}}
\newcommand{\ErrorTok}[1]{\textcolor[rgb]{0.68,0.00,0.00}{#1}}
\newcommand{\ExtensionTok}[1]{\textcolor[rgb]{0.00,0.23,0.31}{#1}}
\newcommand{\FloatTok}[1]{\textcolor[rgb]{0.68,0.00,0.00}{#1}}
\newcommand{\FunctionTok}[1]{\textcolor[rgb]{0.28,0.35,0.67}{#1}}
\newcommand{\ImportTok}[1]{\textcolor[rgb]{0.00,0.46,0.62}{#1}}
\newcommand{\InformationTok}[1]{\textcolor[rgb]{0.37,0.37,0.37}{#1}}
\newcommand{\KeywordTok}[1]{\textcolor[rgb]{0.00,0.23,0.31}{#1}}
\newcommand{\NormalTok}[1]{\textcolor[rgb]{0.00,0.23,0.31}{#1}}
\newcommand{\OperatorTok}[1]{\textcolor[rgb]{0.37,0.37,0.37}{#1}}
\newcommand{\OtherTok}[1]{\textcolor[rgb]{0.00,0.23,0.31}{#1}}
\newcommand{\PreprocessorTok}[1]{\textcolor[rgb]{0.68,0.00,0.00}{#1}}
\newcommand{\RegionMarkerTok}[1]{\textcolor[rgb]{0.00,0.23,0.31}{#1}}
\newcommand{\SpecialCharTok}[1]{\textcolor[rgb]{0.37,0.37,0.37}{#1}}
\newcommand{\SpecialStringTok}[1]{\textcolor[rgb]{0.13,0.47,0.30}{#1}}
\newcommand{\StringTok}[1]{\textcolor[rgb]{0.13,0.47,0.30}{#1}}
\newcommand{\VariableTok}[1]{\textcolor[rgb]{0.07,0.07,0.07}{#1}}
\newcommand{\VerbatimStringTok}[1]{\textcolor[rgb]{0.13,0.47,0.30}{#1}}
\newcommand{\WarningTok}[1]{\textcolor[rgb]{0.37,0.37,0.37}{\textit{#1}}}

\providecommand{\tightlist}{%
  \setlength{\itemsep}{0pt}\setlength{\parskip}{0pt}}\usepackage{longtable,booktabs,array}
\usepackage{calc} % for calculating minipage widths
% Correct order of tables after \paragraph or \subparagraph
\usepackage{etoolbox}
\makeatletter
\patchcmd\longtable{\par}{\if@noskipsec\mbox{}\fi\par}{}{}
\makeatother
% Allow footnotes in longtable head/foot
\IfFileExists{footnotehyper.sty}{\usepackage{footnotehyper}}{\usepackage{footnote}}
\makesavenoteenv{longtable}
\usepackage{graphicx}
\makeatletter
\def\maxwidth{\ifdim\Gin@nat@width>\linewidth\linewidth\else\Gin@nat@width\fi}
\def\maxheight{\ifdim\Gin@nat@height>\textheight\textheight\else\Gin@nat@height\fi}
\makeatother
% Scale images if necessary, so that they will not overflow the page
% margins by default, and it is still possible to overwrite the defaults
% using explicit options in \includegraphics[width, height, ...]{}
\setkeys{Gin}{width=\maxwidth,height=\maxheight,keepaspectratio}
% Set default figure placement to htbp
\makeatletter
\def\fps@figure{htbp}
\makeatother
\newlength{\cslhangindent}
\setlength{\cslhangindent}{1.5em}
\newlength{\csllabelwidth}
\setlength{\csllabelwidth}{3em}
\newlength{\cslentryspacingunit} % times entry-spacing
\setlength{\cslentryspacingunit}{\parskip}
\newenvironment{CSLReferences}[2] % #1 hanging-ident, #2 entry spacing
 {% don't indent paragraphs
  \setlength{\parindent}{0pt}
  % turn on hanging indent if param 1 is 1
  \ifodd #1
  \let\oldpar\par
  \def\par{\hangindent=\cslhangindent\oldpar}
  \fi
  % set entry spacing
  \setlength{\parskip}{#2\cslentryspacingunit}
 }%
 {}
\usepackage{calc}
\newcommand{\CSLBlock}[1]{#1\hfill\break}
\newcommand{\CSLLeftMargin}[1]{\parbox[t]{\csllabelwidth}{#1}}
\newcommand{\CSLRightInline}[1]{\parbox[t]{\linewidth - \csllabelwidth}{#1}\break}
\newcommand{\CSLIndent}[1]{\hspace{\cslhangindent}#1}

\makeatletter
\makeatother
\makeatletter
\makeatother
\makeatletter
\@ifpackageloaded{caption}{}{\usepackage{caption}}
\AtBeginDocument{%
\ifdefined\contentsname
  \renewcommand*\contentsname{Table of contents}
\else
  \newcommand\contentsname{Table of contents}
\fi
\ifdefined\listfigurename
  \renewcommand*\listfigurename{List of Figures}
\else
  \newcommand\listfigurename{List of Figures}
\fi
\ifdefined\listtablename
  \renewcommand*\listtablename{List of Tables}
\else
  \newcommand\listtablename{List of Tables}
\fi
\ifdefined\figurename
  \renewcommand*\figurename{Figure}
\else
  \newcommand\figurename{Figure}
\fi
\ifdefined\tablename
  \renewcommand*\tablename{Table}
\else
  \newcommand\tablename{Table}
\fi
}
\@ifpackageloaded{float}{}{\usepackage{float}}
\floatstyle{ruled}
\@ifundefined{c@chapter}{\newfloat{codelisting}{h}{lop}}{\newfloat{codelisting}{h}{lop}[chapter]}
\floatname{codelisting}{Listing}
\newcommand*\listoflistings{\listof{codelisting}{List of Listings}}
\makeatother
\makeatletter
\@ifpackageloaded{caption}{}{\usepackage{caption}}
\@ifpackageloaded{subcaption}{}{\usepackage{subcaption}}
\makeatother
\makeatletter
\@ifpackageloaded{tcolorbox}{}{\usepackage[many]{tcolorbox}}
\makeatother
\makeatletter
\@ifundefined{shadecolor}{\definecolor{shadecolor}{rgb}{.97, .97, .97}}
\makeatother
\makeatletter
\makeatother
\journal{Journal Name}
\ifLuaTeX
  \usepackage{selnolig}  % disable illegal ligatures
\fi
\usepackage[]{natbib}
\bibliographystyle{elsarticle-num}
\IfFileExists{bookmark.sty}{\usepackage{bookmark}}{\usepackage{hyperref}}
\IfFileExists{xurl.sty}{\usepackage{xurl}}{} % add URL line breaks if available
\urlstyle{same} % disable monospaced font for URLs
\hypersetup{
  pdftitle={Research Article},
  pdfauthor={Joey W. Trampush; Biggie Smalls},
  pdfkeywords={keyword1, keyword2, keyword3},
  colorlinks=true,
  linkcolor={blue},
  filecolor={Maroon},
  citecolor={Blue},
  urlcolor={Blue},
  pdfcreator={LaTeX via pandoc}}

\setlength{\parindent}{6pt}
\begin{document}

\begin{frontmatter}
\title{Research Article \\\large{A Short Subtitle} }
\author[1]{Joey W. Trampush%
\corref{cor1}%
\fnref{fn1}}
 \ead{joey.trampush@med.usc.edu} 
\author[2]{Biggie Smalls%
%
\fnref{fn2}}
 \ead{biggyy@usc.edu} 

\affiliation[1]{organization={USC Keck School of
Medicine, Psychiatry},addressline={Street Address},city={Los
Angeles},postcode={90033},postcodesep={}}
\affiliation[2]{organization={USC, Department of Psychiatry and the
Behavioral Sciences},addressline={Street
Address},city={City},postcode={Postal Code},postcodesep={}}

\cortext[cor1]{Corresponding author}
\fntext[fn1]{This is the first author footnote.}
\fntext[fn2]{Another author footnote, this is a very long footnote and
it should be a really long footnote. But this footnote is not yet
sufficiently long enough to make two lines of footnote text.}
        





\begin{keyword}
    keyword1 \sep keyword2 \sep 
    keyword3
\end{keyword}
\end{frontmatter}
    \ifdefined\Shaded\renewenvironment{Shaded}{\begin{tcolorbox}[interior hidden, boxrule=0pt, frame hidden, breakable, enhanced, borderline west={3pt}{0pt}{shadecolor}, sharp corners]}{\end{tcolorbox}}\fi

\renewcommand*\contentsname{Table of contents}
{
\hypersetup{linkcolor=}
\setcounter{tocdepth}{3}
\tableofcontents
}
\textbf{Keywords:} keyword1, keyword2, keyword3

\textbf{Highlights:} These are the highlights.

\hypertarget{abstract}{%
\subsection{Abstract}\label{abstract}}

Consequat id Lorem consectetur ipsum sit pariatur excepteur officia esse
minim quis. Consectetur minim duis non fugiat. Cupidatat sunt quis
veniam minim enim qui qui excepteur cupidatat voluptate culpa. Veniam
velit nostrud fugiat cupidatat est ad minim enim ad culpa officia qui eu
aliqua mollit. Dolor occaecat labore fugiat. Deserunt velit deserunt
quis magna voluptate consectetur. Tempor cupidatat eiusmod commodo
magna. Deserunt consectetur nulla officia occaecat. Non proident minim
ea quis elit occaecat ipsum adipisicing elit pariatur. Sit veniam
adipisicing sint Lorem minim sit.

\newpage{}

\hypertarget{introduction}{%
\subsection{Introduction}\label{introduction}}

Excepteur eiusmod dolore qui labore sit adipisicing irure enim tempor do
id eu magna ea. Mollit in Lorem duis amet veniam do laboris ipsum. Minim
esse duis voluptate esse ea et proident. Ut est dolor Lorem mollit culpa
tempor proident qui eu qui enim nisi proident proident nisi. Eiusmod ut
laboris id cupidatat minim velit qui sunt velit anim veniam labore
laborum eu. Veniam aliqua consequat anim do laboris amet nulla nulla
commodo ut officia nulla. Velit sunt ea ea fugiat cupidatat irure
cupidatat fugiat anim cillum. Culpa est cillum aliquip mollit aliquip
occaecat veniam fugiat anim est. Non enim amet adipisicing aliqua enim
enim mollit dolore cupidatat incididunt sunt cupidatat. Laboris qui enim
id reprehenderit reprehenderit sint amet pariatur deserunt magna
excepteur ipsum laborum voluptate incididunt.

\newpage{}

\hypertarget{methods}{%
\subsection{Methods}\label{methods}}

Velit amet aute elit. Consectetur do est velit ullamco reprehenderit
magna adipisicing minim voluptate veniam fugiat. Nostrud aliquip laborum
nostrud Lorem dolore. Qui ex duis ullamco minim ea incididunt pariatur.
Quis adipisicing deserunt eiusmod officia amet ullamco. Proident esse
tempor in aliqua dolor pariatur duis qui. Cupidatat dolor velit et
laboris ipsum elit voluptate velit consequat consectetur quis id duis
nisi quis. Velit eiusmod consequat laboris aliquip ex id et id ipsum
excepteur esse. Elit officia laboris exercitation labore ut eiusmod
mollit do veniam et et sint ex. Laborum cupidatat laboris esse do
adipisicing id proident eiusmod esse eiusmod ipsum amet reprehenderit
sint.

\newpage{}

\hypertarget{results}{%
\subsection{Results}\label{results}}

Quis irure nulla nulla adipisicing eu fugiat dolore ut nostrud qui
laborum ad consequat minim. Cupidatat aliqua consequat exercitation elit
ad fugiat esse sunt quis dolore sit excepteur sunt sit. Enim et ut sit
voluptate sit sunt adipisicing adipisicing magna sunt. Est dolor do sit
occaecat labore adipisicing deserunt. Labore ad velit cillum. Cupidatat
nulla ipsum elit in occaecat cupidatat veniam. Laboris ipsum duis
proident dolor. Veniam quis aute culpa dolore. Non proident ea ullamco
nisi dolor fugiat laborum sint exercitation nulla velit. Irure eu ad
culpa tempor nisi sunt ad et culpa sunt officia tempor exercitation
reprehenderit ullamco.

\newpage{}

Here is an example of inline code in the middle of a sentence.

\hypertarget{discussion}{%
\subsection{Discussion}\label{discussion}}

Eiusmod adipisicing occaecat minim reprehenderit exercitation id
adipisicing aute incididunt. Incididunt consequat exercitation fugiat.
Aliquip amet mollit elit velit et eiusmod culpa nostrud et. Dolore
excepteur veniam consequat officia commodo cupidatat qui dolore qui est
consequat occaecat amet enim anim. Do exercitation cupidatat in
consequat sit est ad cillum deserunt labore consequat ipsum consectetur.
Fugiat et in excepteur sit ex Lorem ut ullamco excepteur amet culpa
Lorem excepteur eiusmod ut. Enim est ea ex eiusmod qui sit amet ad
aliqua enim amet. Excepteur ullamco ullamco fugiat eiusmod ut culpa enim
commodo anim ex laborum velit exercitation voluptate. Est culpa minim
laboris voluptate qui Lorem dolore. Ipsum Lorem cillum ullamco et
commodo.

\newpage{}

\hypertarget{acknowledgements}{%
\subsection{Acknowledgements}\label{acknowledgements}}

\newpage{}

\newpage{}

\hypertarget{references}{%
\subsection{References}\label{references}}

\hypertarget{refs}{}
\begin{CSLReferences}{0}{0}
\end{CSLReferences}

\newpage{}

\hypertarget{colophon}{%
\subsubsection{Colophon}\label{colophon}}

This report was generated on 2023-02-15 16:25:56 using the following
computational environment and dependencies:

\begin{Shaded}
\begin{Highlighting}[]
\CommentTok{\# which R packages and versions?}
\ControlFlowTok{if}\NormalTok{ (}\StringTok{"devtools"} \SpecialCharTok{\%in\%} \FunctionTok{installed.packages}\NormalTok{()) devtools}\SpecialCharTok{::}\FunctionTok{session\_info}\NormalTok{()}
\end{Highlighting}
\end{Shaded}

\begin{verbatim}
- Session info ---------------------------------------------------------------
 setting  value
 version  R version 4.2.2 Patched (2023-02-09 r83797)
 os       macOS Big Sur ... 10.16
 system   x86_64, darwin17.0
 ui       X11
 language (EN)
 collate  en_US.UTF-8
 ctype    en_US.UTF-8
 tz       America/Los_Angeles
 date     2023-02-15
 pandoc   3.1 @ /usr/local/bin/ (via rmarkdown)

- Packages -------------------------------------------------------------------
 ! package     * version    date (UTC) lib source
 P cachem        1.0.6.9000 2022-06-03 [?] https://r~
 P callr         3.7.3.9000 2022-11-03 [?] https://r-lib.r-universe.dev (R 4.2.2)
 P cli           3.6.0.9000 2023-01-09 [?] https://r-lib.r-universe.dev (R 4.2.2)
 P crayon        1.5.2      2022-09-29 [?] CRAN (R 4.2.0)
 P devtools      2.4.5.9000 2022-10-11 [?] https://r-lib.r-universe.dev (R 4.2.1)
 P digest        0.6.31     2022-12-11 [?] CRAN (R 4.2.0)
 P ellipsis      0.3.2.9000 2022-06-23 [?] https://r-lib.r-universe.dev (R 4.2.0)
 P evaluate      0.20.1     2023-01-17 [?] https://r-lib.r-universe.dev (R 4.2.2)
 P fastmap       1.1.0.9000 2022-05-15 [?] https://r~
 P fs            1.6.1.9000 2023-02-08 [?] https://r-lib.r-universe.dev (R 4.2.2)
 P glue          1.6.2.9000 2023-01-26 [?] https://t~
 P htmltools     0.5.4.9000 2022-12-07 [?] https://rstudio.r-universe.dev (R 4.2.2)
 P htmlwidgets   1.6.1      2023-01-07 [?] CRAN (R 4.2.0)
 P httpuv        1.6.9.9000 2023-02-14 [?] https://rstudio.r-universe.dev (R 4.2.2)
 P jsonlite      1.8.4      2022-12-06 [?] CRAN (R 4.2.0)
 P knitr         1.42.2     2023-02-13 [?] https://yihui.r-universe.dev (R 4.2.2)
 P later         1.3.0.9000 2022-05-15 [?] https://r~
 P lifecycle     1.0.3.9000 2022-10-07 [?] https://r-lib.r-universe.dev (R 4.2.1)
 P magrittr      2.0.3.9000 2022-05-29 [?] https://tidyverse.r-universe.dev (R 4.2.0)
 P memoise       2.0.1.9000 2022-05-28 [?] https://r~
 P mime          0.12.1     2022-06-25 [?] https://yihui.r-universe.dev (R 4.2.1)
 P miniUI        0.1.1.1    2018-05-18 [?] CRAN (R 4.2.0)
 P pkgbuild      1.4.0.9000 2023-01-06 [?] https://r~
 P pkgload       1.3.2.9000 2022-11-16 [?] https://r-lib.r-universe.dev (R 4.2.2)
 P prettyunits   1.1.1.9000 2022-05-10 [?] https://r~
 P processx      3.8.0.9000 2022-12-18 [?] https://r~
 P profvis       0.3.7.9000 2022-04-27 [?] https://rstudio.r-universe.dev (R 4.2.0)
 P promises      1.2.0.9000 2022-04-28 [?] https://rstudio.r-universe.dev (R 4.2.0)
 P ps            1.7.2.9000 2022-10-27 [?] https://r-lib.r-universe.dev (R 4.2.1)
 P purrr         1.0.1.9000 2023-01-10 [?] https://tidyverse.r-universe.dev (R 4.2.2)
 P R6            2.5.1.9000 2022-12-27 [?] https://r-lib.r-universe.dev (R 4.2.2)
 P Rcpp          1.0.10     2023-01-22 [?] CRAN (R 4.2.0)
 P remotes       2.4.2.9000 2023-02-04 [?] https://r-lib.r-universe.dev (R 4.2.2)
   renv          0.16.0-65  2023-02-09 [1] https://rstudio.r-universe.dev (R 4.2.2)
 P rlang         1.0.6.9000 2022-10-05 [?] https://r-lib.r-universe.dev (R 4.2.1)
 P rmarkdown     2.20.1     2023-01-20 [?] https://rstudio.r-universe.dev (R 4.2.2)
 P sessioninfo   1.2.2.9000 2022-05-14 [?] https://r~
 P shiny         1.7.4.9002 2023-01-25 [?] https://rstudio.r-universe.dev (R 4.2.2)
 P stringi       1.7.12     2023-01-11 [?] CRAN (R 4.2.2)
 P stringr       1.5.0.9000 2022-12-07 [?] https://tidyverse.r-universe.dev (R 4.2.2)
 P urlchecker    1.0.1.9000 2022-05-14 [?] https://r~
 P usethis       2.1.6.9000 2023-02-14 [?] Github (r-lib/usethis@18c9179)
 P vctrs         0.5.2.9000 2023-01-25 [?] https://r-lib.r-universe.dev (R 4.2.2)
 P xfun          0.37.1     2023-01-31 [?] https://yihui.r-universe.dev (R 4.2.2)
 P xtable        1.8-6      2022-04-14 [?] https://r-forge.r-universe.dev (R 4.2.0)
 P yaml          2.3.7      2023-01-23 [?] CRAN (R 4.2.0)

 [1] /Users/joey/.renv/library/pj.research.compendium.root-717ea691/R-4.2/x86_64-apple-darwin17.0
 [2] /Users/joey/Dropbox/pfc/pj.research.compendium.root/renv/sandbox/R-4.2/x86_64-apple-darwin17.0/84ba8b13

 P -- Loaded and on-disk path mismatch.

------------------------------------------------------------------------------
\end{verbatim}

The current Git commit details are:

\begin{Shaded}
\begin{Highlighting}[]
\CommentTok{\# what commit is this file at? }
\ControlFlowTok{if}\NormalTok{ (}\StringTok{"git2r"} \SpecialCharTok{\%in\%} \FunctionTok{installed.packages}\NormalTok{() }\SpecialCharTok{\&}\NormalTok{ git2r}\SpecialCharTok{::}\FunctionTok{in\_repository}\NormalTok{(}\AttributeTok{path =} \StringTok{"."}\NormalTok{)) git2r}\SpecialCharTok{::}\FunctionTok{repository}\NormalTok{(here}\SpecialCharTok{::}\FunctionTok{here}\NormalTok{())  }
\end{Highlighting}
\end{Shaded}

\begin{verbatim}
Local:    main /Users/joey/Dropbox/pfc/pj.research.compendium.root
Remote:   main @ origin (https://github.com/brainworkup/pj.research.compendium.root.git)
Head:     [29960b5] 2023-02-16: updated readme
\end{verbatim}


  \bibliography{analysis/paper/references.bib}


\end{document}
